\chapter{Summary}
This chapter gives a brief conclusion of the results and an outlook.

\section{Conclusion}

The RRT* algorithm is proved to be highly suitable for finding paths between two points. In comparison with other methods, it is adaptable to high-dimensional spaces. However, to generate feasible path in large spaces and to get optimal results, long calculation time is required, which is a major drawback of this method. Otherwise, paths, like the one for object scanning (see Section \ref{sec:object_scanning}), would result in a zig-zag-pattern. Although \textit{Scubo} is able to handle such kind of trajectories, they are not pleasing in an efficiency matter.

\section{Outlook}

In a further step, the generated paths need to be smoothed in a post-processing step in order to remove the zig-zag-pattern. Furthermore, the configuration space can be expanded in order to produce a trajectory consisting of position and orientation information. An implementation in C++/ROS and the combination with a trajectory controller is required to test the path planner on the real system. \\

The path planner developed in this bachelor thesis requires information of the static map in order to produce feasible paths from a fixed starting position. However, this rarely the case in real world application. In addition to static, there are also dynamic obstacles. Therefore, an real-time RRT* algorithm in combination with SLAM must be developed in order to recognize moving obstacles and to locally re-plan the path. Additionally, an object identification algorithm must be produced in order to autonomously plan scanning routes. 

   