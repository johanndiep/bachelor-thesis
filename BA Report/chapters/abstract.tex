\chapter*{Abstract}
\addcontentsline{toc}{chapter}{Abstract}
%\chapter*{Zusammenfassung}
%\addcontentsline{toc}{chapter}{Zusammenfassung}

\textsc{Scubo} is a submersible ROV (Remotely Operated Vehicle) developed in the context of a focus project at ETH Zurich. Due to the arrangement of eight non-moving propulsion units and the superposition of its forces, it is able to move in any direction without prior change of its orientation. Besides, it is equipped with six cameras pointing in each direction and the necessary hardware for autonomy. Several bachelor theses are concerned with the software implementation of an autonomous path following system. \\  

This bachelor thesis discusses the implementation of an offline global path planning algorithm for the \textsc{Scubo} robot. The algorithm uses a point cloud\footnote{A point cloud is a set of data points representing positions of features. It can be generated by post-processing recordings of a stereo camera and an IMU with a mapping software.} of the environment to generate a set of collision-free waypoints between an initial and a goal state for the trajectory controller to follow. Since the robot is capable of omnidirectional movements, the generated path does not have to satisfy any kinematic or dynamic constraints of the system other than avoiding collision with obstacles. \\

At first, the basic idea behind available path planning methods are explained on a conceptual level. In a second step, the implementation of the path planning algorithm using a variant of the rapidly-exploring random tree (RRT) method is presented in more detail. Lastly, the algorithm is used in a practical application where the goal is to scan an object from all sides under different angles. In this case, the developed path planner in combination with a traveling salesman problem solver proves to be highly suitable to generate a collision-free and cost-efficient path around an object.   


